\chapter{準備}
%群, 環, 体について書く. 
\section{群の定義} 
集合$G$の直積集合$G \times G$から集合$G$への写像が1つ与えられているとき,\ この写像を$G$の2項演算と呼ぶ. 2項演算が与えられ, 次の条件全てを満足する場合$G$はこの演算に関して群という.\
\begin{enumerate}
 \item 結合律\\
 ${}^{\forall}a,b,c \in G$に対して常に,$(a \ast b) \ast c=a \ast (b \ast c)$が成立する.
 \item 単位元の存在\\
 $e \in G$,\ ${}^{\forall}a \in G$に対して$a \ast e=e \ast a=a$が成立する.
 \item 逆元の存在\\ 
 $G$に属する任意の元$a$に対して$a \ast b=b \ast a=e$となる元$b$が存在する.
\end{enumerate} 
2項演算が1. のみを満たす$G$と$*$の組$(G, \ast )$はこの2項演算に関する半群という.\\
更に,\ 群$G$が\\
\quad \ 4.交換法則\\
$a,b \in G$に対し,$a \ast b=b \ast a$
を満たすとき,\ 群$G$は可換群であるという.\\


集合$G$が2項演算$\ast$に対して可換群であるとき$\ast$が, $+$で表される場合その群を加法群と呼ぶ.\ そのとき $x+y$を$x$と$y$の和といい, 単位元を0,\ $x$の逆元を$-x$で表す.\\
群$(G,\ast)$において2項演算が明らかな場合単に群$G$ということもある.\\
また, 群$G$に属する元の個数が有限であるとき$G$を有限群,\ そうでないとき無限群という.\\


可換群$G$の任意の元が1つの元$a$のべき乗で表せるとき,\ $G$を$a$で生成された巡回群といい,\ 
$G=\langle a \rangle$で表す.\ $aを生成元,\ あるいは原始元という.\ $
\newpage
\section{環の定義}
\subsection{環}
2種類の2項演算(加法$+$と乗法$\cdot$\ )の定義された集合$R$が次の条件を満足するとき, $(R,+,\cdot)$は環であるいう.\\
\begin{enumerate}
 \item 加法に関して可換群をなす.\\
 $(a+b)+c=a+(b+c)$\\
 $a+b=b+a$\\
 $0+a=a+0$\\
 $(-a)+a=a+(-a)=0$\\
 \item 乗法に関して半群をなし乗法に関する単位元が存在する.\\
 $a \in R$に対して$a \cdot e=e \cdot a=a$となる$e \in R$が存在する.\\
 \item 分配法則\\
 $a,b,c \in R$に対して,\\
 $a \cdot (b+c)=a \cdot b+a \cdot c$\\
 $(a+b) \cdot c=a \cdot c+b \cdot c$\\
 が成立する.\ 
\end{enumerate}

更に, 環 $R$ において,\\
\quad \ 4.交換法則\\
\quad \quad \ \ $a,b \in R$ に対し,\ $a \cdot b=b \cdot a$ \\
を満たすとき, 群$R$は可換環であるといい, そうでないときを非可換環という.\\
環における単位元は加法単位元$0_{R}$と, 乗法単位元$1_{R}$がある.\
環の乗法の記号$\cdot$は省略されることが多い.\ すなわち,\ 
$x \cdot y$は$xy$と書かれる.\ 以下この記法で書くことにする.

\subsection{部分環,\ イデアル,\ 商環}
環$R$の部分集合でそれ自身が$R$の演算において,\ 環になるものを$R$の部分環という. 


環$R$の部分集合$I$において,\
\begin{enumerate}
 \item $a,b \in I\ ならばa+b \in I$
 \item $a \in I\ とr \in R\ に対してra \in I$
 \item $a \in I\ とr \in R\ に対してar \in I$
\end{enumerate}
条件1.\ と2.\ を満たすとき,\ $IはR$の左イデアル,\ 条件1.\ と3.\ を満たすとき,\ $I$は
$R$の右イデアル,\ 条件1.\ から3.\ まで全てを満足するとき,\ $IはRの$両側イデアルまたは
単にイデアルという.\ $R$が可換環の場合は左イデアル,\ 右イデアル,\ 両側イデアルは一致
する.


環$R$の中のイデアル$I$の生成元の集合とは,\ $I$の元の集合であって,\ $I$の任意の元がその
集合の元の$R$係数の有限な1次結合であるようなものである.\ イデアルはもし生成元の有限集合
をもつなら,\ 有限生成といわれる.\ $I$が元の集合$\{f_1,\cdot\cdot\cdot,f_l\} \subset I$
によって生成されるなら,\ $\displaystyle I=\sum_{i=1}^{l}Rf_i$,\ または単に$I=(f_1,\cdot\cdot\cdot,f_l)$
と書く.\ 


ここで, 整数$a,\ b$の差$a-b$が自然数$n$で割り切れるとき, $a,\ bは法nに関して合同であるといい,\ a \equiv b \ \mbox{mod} \ n$と表現する. 互いに合同な整数全体の集合を剰余類という. 
環$R$の元$x,\ y$がイデアル$I$を法として合同であるとは,\ $x+i=y$となる元$i \in I$が存在することであり,\ このとき\ $x \equiv y \pmod{I}$と書く.\ この関係は同値関係である.\ 環$R$の
イデアル$I$による同値類を$[x]$と書けば,\ $[x]=x+I=\{x+i|i \in I\}となり,\ 同値類の集合R/I$
に加法と乗法,\ つまり, 
\begin{eqnarray}
 [x]+[y]=[x+y] \ ,\ \ [x]\cdot[y]=[xy]
\end{eqnarray}
が定義できる.\ これらの演算に関して同値類の集合$R/I$は環になり,\ $Iを法とするRの商環または剰余環$
という. 

\bigskip


\subsection{多項式環}
可換環$R$において,\ $x$を不定元(変数)としたとき,\ $R上の多項式の集合$は,\ 
\begin{eqnarray}
\{a_nx^n+ \cdot\cdot\cdot+a_1x+a_0\ |\ a_0,\ a_1,\ \cdot\cdot\cdot,\ a_n\ \in R,\ nは0か正の整数\}
\end{eqnarray}
と定義される.\ $f(x)=b_nx^n+ \cdot\cdot\cdot+b_1x+b_0がR上の多項式$で$b_n \neq 0$としたとき,\ $n$を多項式
$f$の次数といい,\ $\deg f$と表す. 特に,\ $n=0$のとき,\ $f(x)=b_0 \in R$となるが,\ これを定数と呼ぶ.\ 
$0 \in R$の次数は$-\infty$とする.\ また,\ 最高次の係数が1である多項式をモニック多項式という.\ 


$xを不定元とする可換環R$上の多項式全体の集合には,\ $R$における2項演算を用いて,\ 次のように
2項演算を定義することができる.\ $R$上の2つの多項式$f(x)=b_nx^n+ \cdot\cdot\cdot+b_1x+b_0,\ g(x)=
c_mx^m+ \cdot\cdot\cdot+c_1x+c_0$に対して,\ 
\begin{eqnarray}
f(x)+g(x)=\sum_{k\geq0}(b_k+c_k)x^k
\end{eqnarray}
\begin{eqnarray}
f(x)\cdot g(x)=\sum_{k=0}^{n+m}(\sum_{i+j=k}b_ic_j)x^k
\end{eqnarray}
と2つの2項演算$+と\ \cdot$を定めると,\ これに関して,\ $x$を不定元とする$R$上の多項式
の全体集合は可換環になる.\ ただし,\ $x^0=1,\ 0\cdot x=0\ (0,\ 1\in R)$と定める.\ 
こうして得られた環を,\ $R$上の多項式環と呼び,\ $R[x]$と表す.


可換環$R$上の多項式$f(x)$が,\ 1次以上の多項式$g(x),\ h(x) \in R[x]$によって,\ $f(x)=g(x)h(x)$
となるとき,\ $g(x)|f(x),\ h(x)|f(x)$と表し,\ $g(x),\ h(x)をf(x)の因子と呼ぶ$.\ $f(x) \in R[x]$
が因子を持たないとき,\ $f(x)はR$上既約であるといわれる.\ $f(x)$が既約でないとき,\ 可約であると
いう.\ 


$TをT \supset R$であり,\ $Rで定義されている2項演算$に対して,\ 環になっているとする.\ このとき,\ 
不定元$xにTの元t$を代入することにより,\ 
\begin{eqnarray}
f(t)=b_nt^n+ \cdot\cdot\cdot +b_1t+b_0 \in T,\ b_i \in R
\end{eqnarray}
が得られる.\ $f(t)=0 \in R$となるときの$t$を,\ $f(x)$の零点という.

\newpage
\section{体の定義}
集合$\mathbb {F}$が次の条件を満たすとき,$\mathbb {F}$は体である.\\
集合$\mathbb {F}$に対して加法と乗法が定義されているとする.
\begin{enumerate}
\item 環である.
\item 0以外の$\mathbb {F}$における全ての元には乗法に関し逆元が存在する.
\end{enumerate}
このとき,環が可換環であるならば可換体という.\\
もし,$\mathbb {F}$において単位元1をそれ自身に加えていっても決して0にならないならば,$\mathbb {F}$の標数は0であるといい,char($\mathbb {F}$)=0と書く.この場合,$\mathbb {F}$に含まれる最小の体は有理数体$\mathbb {Q}$である.そうでない場合,$1+1+\cdots +1(p$回)が0に等しいような素数$p$があり,$\mathbb {F}$の標数は$p$であるといい,char$(\mathbb {F})=p$と書く.この場合,$\mathbb {F}$に含まれる最小の体は$\mathbb {F}_p=\mathbb {Z}/p\mathbb {Z}$ である.この様な最小の体を$\mathbb {F}$の素体という.\\

\par
\subsection{有限体,ガロア体}
\par

体$\mathbb {F}$の中で,有限個の元からなるものを有限体あるいはガロア体という.無限個の元からなる体を無限体という.元の数が$q$の有限体を$\mathbb {F}_q$で表す.ガロア体$\mathbb {F}_q$は元の数$q$が素数$p$あるいは素数のべき乗$p^m$のときに限り存在する.体$\mathbb {F}$の元を係数とする多項式を,$\mathbb {F}$上の多項式という.$\mathbb {F}_q$上における多項式間の係数演算は$\mathbb {F}_q$の演算である.なお,有限体の中で最も小さい体は0と1の二つの元からなる体$\mathbb {F}_2$である.\\

\par
\subsection{拡大体}
\par

$L$を体,$K$を$L$の部分体としたとき,$L$を$K$の拡大体という.さらに$M$が$L$の部分体であり,$K$の拡大体であるとき,$M$を$L$と$K$の中間体という.\\
$\mathbb {F}$を含む拡大体$\mathbb {K}$の元$\alpha $は,もし$f(\alpha )=0$である1変数多項式$f(X)\in \mathbb {F}[X]$があるなら,$\mathbb {F}$上代数的であるという.\\
この場合,$\mathbb {F}[X]$の中に$\alpha $が根であるmonic既約多項式が一意に存在する.そして$\alpha $が満たす他のどんな多項式も,このmonic多項式で割ることができなければならない.このmonic既約多項式は,$\alpha $の最小多項式と呼ばれる.もし,$\alpha $の最小多項式が次数$d$を持つならば,$\mathbb {F}(\alpha )$における任意の元,すなわち$\alpha $のべきと$\mathbb {F}$の元に関する任意の有理式は,べき$1,\alpha , \alpha ^2,\cdots ,\alpha ^{d-1}$の1次結合として表現できる.\\
したがって,これらの$\alpha $のべきは$\mathbb {F}$上における$\mathbb {F}(\alpha )$の基底をなす.よって,$\alpha $を付加することにより得られる拡大次数は$\alpha $の最小多項式における次数と同じである.\\

\par
\subsection{代数的閉包}
\par
体$\mathbb {F}$に係数を持つ全ての多項式が,1次因子に完全に分解するという性質を持つならば,$\mathbb {F}$は代数的に閉じているという.同じことであるが,$\mathbb {F}$に係数を持つ全ての多項式が$\mathbb {F}$に根を持つことを要求すれば充分である.代数的に閉じている最小の$\mathbb {F}$の拡大体は,$\mathbb {F}$の代数的閉包といい,$\mathbb {\overline{F}}$と表す.\\

\par


\section{離散対数問題とElGamal\ 暗号}
公開鍵暗号とは,\ 暗号化鍵は公開し, 誰もが使えるようにしておくが,\ 復号に使う鍵は秘密にする
暗号である.\ 公開鍵から秘密鍵を求めることは困難なので,\ 暗号文の正規の受信者以外は暗号文を
解読できないという原理になっている.


公開鍵暗号の一例としてElGamal暗号が挙げられる.\ ElGamal暗号は,\ 離散対数問題という問題の困難さに
基づく公開鍵暗号である.\ まず,\ 離散対数問題の定義を述べ,\ その後にElGamal暗号の暗号化と復号
アルゴリズムを述べる.\ 

\subsection{離散対数問題}
群$G$における$g \in G$に対する離散対数問題とは,\ $y \in G$が与えられるとき,\ $g^x = y\ (演算を加法的に書くとxg = y)$である整数$x$が存在するとしたとき,\ それを求めるという問題のことである.\ この$x$を$y$の離散対数という.\ 

\subsection{ElGamal\ 暗号}
使う群は,\ 大きな素数を$p$として,\ $\mathbb{Z}_pの乗法群\mathbb{Z}_{p}^{*} = \{1,2,\cdot\cdot\cdot,p-1\}$
である.\ まず最初に,\ 受信者は$\mathbb{Z}_{p}^{*}の$原始元$g$を選ぶ.\ 次に,\ $\{0,1,\cdot\cdot\cdot,p-2\}$
から$x$をランダムに選び,\ $y = g^x\pmod{p}$を計算する.\ 最後に受信者は,\ $P_K = (p,g,y)$を公開鍵
として公開し,\ 秘密鍵$S_K = x$を秘密に保持する.\ 


次に暗号化であるが,\ 送信者は受信者の公開鍵$(p,g,y)$,\ 平文$m \in \mathbb{Z}_{p}$を入力とし,\ 
暗号文$C = (c_1,c_2)$を,\ $r \in \{0,1,\cdot\cdot\cdot,p-2\}$をランダムに選び,\ $c_1 = g^r\pmod{p}$,\ 
$c_2 = my^r\pmod{p}$として求める.\ これを送信者は受信者に送る.\


最後に復号であるが,\ 受信者は秘密鍵$x$,\ 暗号文$C = (c_1,c_2)$を入力とし,\ 平文$m$を$m = c_2c_{1}^{p-1-x}\pmod{p}$
として求める.\ 
