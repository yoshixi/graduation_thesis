\chapter{序論}
情報化社会の発展により,\ インターネットなどのコンピュータネットワークは
生活に必須のものとなっている.\ しかし,\ 専用の通信路ではないインターネットなど
では,\ その通信路上を流れるデータを誰でも盗み見ることができてしまう.\ そこで,\
他人に見られたくない情報を守る必要がある.\ これを実現し,\ さらには通信者間の認証
なども行えるようにするのが,\ 情報セキュリティ技術である.\ この情報セキュリティ技
術の核となる技術の一つが暗号であり,\ 最近盛んに研究されている.\

\bigskip

\par
楕円曲線暗号とは有限体上の楕円曲線を用いた暗号で, これに対する攻撃方法としてペアリングが用いられた. 暗号設計については, 2000年にJouxによってDiffie-Hellman公開鍵配送方式が三者に拡張され, 境, 大岸, 笠原によって, IDに基づく暗号に適用され, ペアリングの双線形性を用いた暗号プロトコルが数多く提案されている. \par
楕円曲線上の(双線形)ペアリングとは,
楕円曲線の部分群の直積から有限体乗法群への双線形写像であり,
Weil ペアリングやTate ペアリングがその例として知られていてる. しかし, 通常の楕円演算(スカラー倍)に比較して演算量が多いことが問題であった.
したがって, ペアリング演算の高速化が課題となっている. ペアリング演算方法としてMiller Algorithmが一般的に知られている. これを発展させ, supersingular curveにおけるReduced Tate ペアリングの高速化手法を適用したBKLS Algorithmと楕円曲線$y^2 = x^p - x + d$に対して高速化をしたDuursma and Lee Algorithmが提案されている.

\bigskip

\par
既存研究では$k=4,\ 6,\ 8$の楕円曲線に対してペアリング演算の手法であるMiller AlgorithmにDouble-Base Chainsを適用し, 計算コストを算出している. 本論文では$k=2$のordinary curveにおいて, ペアリング演算の手法であるMiller AlgorithmにDouble-Base Chainsを適用した. さらに, $k=2$, 既存研究である$k=4,\ 6$の楕円曲線に対してMAGMAで実装し, 計算時間を求め, 計算コストと比較した.以下に本論文の構成を示す. 第2章で数学的準備, 第3章では楕円曲線について, 点の加算法や楕円離散対数問題について述べる. 第4章でペアリングについて, 第5章でペアリングの計算方法について述べる.第6章で従来手法, 既存研究, 提案手法について, 第7章で実行結果, 第8章で考察と今後の課題を述べる.
