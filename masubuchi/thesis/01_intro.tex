\chapter{序論}
インターネットを代表とするコンピュータネットワーク等の情報通信技術の発展により,インターネットは我々の生活にとって無くてはならない技術となった。その発展により,最近では様々なものに情報技術を組み込む「IoT(Internet of Things)」と呼ばれる技術なども登場し,我々に生活の利便性を向上させている。
しかしその一方で,インターネット上でのクレジットカードの番号を通信する際や,機密性の高い情報を送信する際など,通信される情報が盗み取られ,複製,改ざんされる危険性がある。そのため,第3者に見られたくない情報を守る必要がある。これを実現し,さらには通信している相手が本当に目的の人物なのかを確かめるため,認証なども行うようにするのが情報セキュリティ技術である。この情報セキュリティ技術の核となる技術の一つが暗号であり,世界中で盛んに研究されている。

\bigskip

楕円曲線暗号とは有限体上の楕円曲線を用いた暗号で,これに対する攻撃方法としてペアリングが用いられた。
ペアリングとは楕円曲線上で定義される双線形写像である。2000年以降,暗号プリミティブとして広く利用されるようになった。具体例としてIDを公開鍵として利用可能なIdentity Based Encryption,既存方式より短い署名長で済むShort Signatureなどがあり,従来にない特性を有するプロトコルを構成することが可能である。しかし,主要な暗号要素技術と比較して計算コストが大きく,効率的な演算アルゴリズムが求められている。

\bigskip

ペアリングの演算方法として,Millerアルゴリズムが一般的に知られている。
このアルゴリズムを用いて pairing 演算を実装すると,通常の楕円スカラー倍演算などに比較して演算量が多いため,速度が遅くなることが問題である。このため,Millerのアルゴリズムの高速実装の研究が盛んに行われている。
これらの既存研究としてdistortion mapを用いることで効率化したBKLS Algorithm, 楕円スカラー倍演算の高速化などに使われるwindow法とMillerのアルゴリズムを組み合わせたwindow Miller's Algorithmが提案されている。

本研究では,BKLS Algorithmにwindow法を適用しMiller Alogrithmの高速化を図った。

\bigskip

\par
