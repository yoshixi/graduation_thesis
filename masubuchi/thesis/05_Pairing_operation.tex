\chapter{ペアリング演算の高速化}
%Miller Algorithmの発展, BKLS,window Algorithmについて書く.
\section{Signed Miller Algorithm}
第4章で述べたMiller Algorithmでは部分群の位数$n$を2進展開していた. Signed Miller Algorithmは符号付き2進展開を用いる手法である. 次にそのアルゴリズムを示す.
\par
\begin{table}[htbp]
 \begin{center}
  \begin{tabular}{|l|}
     \hline
     Input: $n, \ P = (x_P,y_P) \in E(\mathbb{F}_q)[n], \ Q = (x_Q,y_Q) \in E(\mathbb{F}_{q^k})$ \\
     Output: $f \in \mathbb{F}_{q^k}$  \\
     \hline
     1: \quad $Q' \in _R E(\mathbb{F}_{q^k})$\\
     2: \quad $S=Q+Q' \in E(\mathbb{F}_{q^k})$\\
     3: \quad $V \gets P, \ f \gets 1,\ f_{-1} = \frac{1}{x_Q - x_P}$\\
     4: \quad $n=\sum^{l - 1}_{i=0} n_i 2^i, \ n_i \in \{-1,0,1\},\ n_0 = 1$\\
     5: \quad for $j \gets l - 1$ down do 0\\
     6: \quad \quad $f \gets f^2 \cdot \frac{g_{V,V}(S)g_{2V}(Q')}{g_{2V}(S)g_{V,V}(Q')}$\\
     7: \quad \quad $V \gets 2V$\\
     8: \quad if $n_j = 1$ then\\
     9: \quad \quad $f \gets f \cdot \frac{g_{V,P}(S)g_{V + P}(Q')}{g_{V + P}(S)g_{V,P}(Q')}$\\
     10: \quad \quad $V \gets V + P$\\
     11: \quad if $n_j = -1$ then\\
     12: \quad \quad $f \gets f \cdot f_{-1} \cdot \frac{g_{V,P}(S)g_{V - P}(Q')}{g_{V - P}(S)g_{V,P}(Q')}$\\
     13: \quad \quad $V \gets V - P$\\
     14: \quad return $f$\\
     \hline
   \end{tabular}
 \end{center}
 \caption{Signed Miller Algorithm}
\end{table}
このように符号付きで展開する方法は後述のアルゴリズムでも同様に用いることができる.
\clearpage
\section{高速化手法}
supersingular curveにおけるReduced Tate ペアリングの高速化手法がBarretoらによっていくつか提案された\cite{BKLS01}. それを次に列挙する.
\subsection{部分体の要素による乗算}
埋め込み次数$k$の因子を$d$とすると, ペアリングの値を変えずに非零な要素$x \in {F}_{q^d}をf_{n,P}(Q)$に乗じることが可能である. $q^k - 1 = (q^d - 1)\sum^{k / d - 1}_{i = 0} q^{id}$と因数分解化のであり, $k > 1のときn | q^k -  1,\ n \not | q^d - 1$となるため, $n | \sum^{k / d - 1}_{i = 0} q^{id}$である. すなわち, $(q^k - 1) / n は因子としてq^d - 1$を必ず含んでいる. よって, $f_{n,P}(Q)は最終べき乗(q^k - 1) / n$の処理を行うことにより, フェルマーの小定理より乗じた値$x^{(q^k - 1) / n} = 1$となる.
\subsection{点による因子の置換}
ペアリングの値は, 引数として与える$Q \in E(K)$とランダムに選択される$R \in E(K)$から構成される因子$D = (Q + R) - (R)$を用いて計算されるが, 因子$Dを点Q$と置換しても正しく計算可能で$P \in E(K_0)[n],\ Q \in E(K)$は線形独立な点とするとき, 次の式が成立する.
\par
\[
e(P,Q) = f_{n,P}(Q)^{q^k-1}
\]
\subsection{分母消去}
Supersingular Curveであれば, $Q'_1 \in E(K_0)$にdistortion mapを適用させることにより, $Q_1 = \psi (Q'_1) \in E(K)$を得る. Ordinary Curveであれば, twist $E'$上の点$Q'_2 \in E(\mathbb{F}_{q^e})からQ'_2 = \psi (Q'_2) \in E(K)を得る$. ここで, $m = \mbox{gcd}(k,d),\ e = k/m$である. よって$Q = (x,y) \in E(K)においてx \in \mathbb{F}_{q^{k/2}} となるとき$, Miller Algorithmの2倍算または加算ステップに出現する分母の要素$g_{2V},g_{V + P}はe(P,Q)$の値を変更することなく省略可能である.
\subsection{群位数のHamming Weight}
Miller Algorithmは部分群の位数$n$に対する2進展開法に基づき, ペアリングの値を計算しているため, 加算ステップの処理回数は$n$のHamming Weight(2進展開した値における1の総数)に依存する. ランダムに選択された$n$の平均的なHamming Weightはビット長の半分程度になるが, 部分群の位数として$n = 2^\alpha \pm 2^\beta \pm 1$を選択することにより, ペアリング演算コストを大幅に抑えることが可能である.
\subsection{最終べき乗の高速化}
Pairing-friendly field $K$における最終べき乗$(q^k - 1) / n$を想定する. $k$次の円分多項式 $\Phi _k(p)$とすれば, $n|q^k - 1$であるため, 次のように変形できる.
\[
f^{(q^k - 1) / n} = (f^{(q^k - 1) / \Phi _k(q)})^{(\Phi _k(q) / n)}
\]
最終べき乗のコストは$(\Phi _k(q) / n)$乗のコストと同等, $(\Phi _k(q) / n)$のビット長は$(\Phi _k(q) / k)\mbox{log} _2 (q^k) - \mbox{log} _2 (n)$となる.
\clearpage
\section{BKLS Algorithm}
前述した高速化手法を用いて, supersingular curveのdistortion map $\psi$ を利用して分母消去の手法を適用したBKLS Algorithm \cite{BKLS02}を次に示す. ordinary curveの場合, $Q' \in E'(K)$として, distortion mapではなくtwistの同型写像 $\psi _d$を用いる.
\vspace{-0.5cm}
\par
\begin{table}[htbp]
 \begin{center}
  \begin{tabular}{|l|}
     \hline
     Input: $P,\ Q \in E(K_0)[n]$\\
     Output: $f \in K$\\
     \hline
     1: \quad $f \gets 1,\ V \gets P$\\
     2: \quad $n=\sum_{l-1}^{i=0} n_i 2^i, \ n_i \in \{0,1\}$\\
     3: \quad for $j \gets l-1$ down 0 do 0\\
     4: \quad \quad $f \gets f^2 \cdot g_{V,\ V}(\psi (Q))$\\
     5: \quad \quad $V \gets 2V$\\
     6: \quad if $n_j = 1$ then\\
     7: \quad \quad $f \gets f \cdot g_{V,\ P}(\psi (Q))$\\
     8: \quad \quad $V \gets V+P$\\
     9: \quad return $f$\\
     \hline
   \end{tabular}
 \end{center}
 \caption{BKLS Algorithm}
\end{table}
\vspace{-0.5cm}
\par

\section{Window Miller Algorithm}
window Miller アルゴリズムは、オンライン事前演算 を用いる方法である。

まず、$i=1,\ldots,2w-1$ に対して2w 個の楕円曲線上の点 $P_i = iP$ を求め、各 $P_i$ において
$i(P) − i(O) = (P _i) − (O) + div(f _i)$ が成り立つような有理式 $f _i$ に対し、
$f' _i  = f' _i(Q + S) /f _i(S)$ と
なる $f' _i ∈ \mathbb{F}_{q^k}$ を求める。
この $P_i, f' _i$ を用いることで、楕円加算および直線 $l _{P, -P_i}$, 垂線 $v _{P_i}$
に関係する演算を削減する。上記アルゴリズムの説明では、簡単にするために w|n を仮定している。
 Miller のアルゴリズムでは m のハミング重み $wH (m)$ 回の楕円加算を行っていたが、window Millerアルゴリズムでは、$n/w$ 回行うことになるため、適切なwを用いれば、楕円加算および直線 $l _{P, -P_i}$, 垂線 $v _{P_i}$ の計算を削減することができる。
\par

\begin{table}[htbp]
 \begin{center}
  \begin{tabular}{|l|}
     \hline
     Input: $n, \ P \in E(\mathbb{F}_q)[n], \ Q \in E(\mathbb{F}_{q^k}) \ S \in E(\mathbb{F}_{q^k})$ \\
     Output: $f \in \mathbb{F}_{q^k}$  \\
     \hline
     (online computation) \\
     1: \quad $P_1 = P, f'_1=1 $\\
     2: \quad for $i \gets 2$ up to do $2^w -1$\\
     3: \quad \quad $P_i \gets P + P_{i-1} $\\
     4: \quad \quad $f'_i \gets f'_{i-1} \cdot \frac{g_{P,-P_i}(S)g_{\mathcal{O},P_i}(Q+S)}{g_{P,-P_i}(S)g_{\mathcal{O},P_i}(Q+S)}$\\

     (main computation) \\
     5: \quad $V \gets P, f \gets 1 $\\
     6: \quad $n=\sum^{l - 1}_{i=0} n_i 2^i, \ n_i \in \{0,1\},\ n_0 = 1$\\

     7: \quad for $ n-1 \gets i$ down to 0 step w\\
     8: \quad step 8-1 から 8-2 をw回繰り返す\\
     8-1: \quad\quad $V \gets 2V $\\
     8-2: \quad \quad $f \gets f^2 \cdot \frac{g_{V,-2V}(Q+S)g_{\mathcal{O},2V}(S)}{g_{V,-2V}(Q+S)g_{\mathcal{O},2V}(S)}$\\

     9: \quad $m' \gets =\sum^{j=i-w+1}_{i} m[j]2^{j-i+w-1} $\\
     10: \quad if $m' \neq 0$ then\\
     10-1: \quad \quad $f \gets ff'_m \cdot \frac{g_{V,P_m}(Q+S)g_{\mathcal{O},V+P_m}(S)}{g_{V,P_m}(Q+S)g_{\mathcal{O},V+P_m}(S)}$ \\
     10-2: \quad \quad $V \gets V + P_{m'} $\\
     11: \quad return $f$\\
     \hline
   \end{tabular}
 \end{center}
 \caption{Window Miller Algorithm}
\end{table}
