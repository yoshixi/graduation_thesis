\chapter{提案手法}
%既存手法であるwindow法, NAF, left-to-right, double-base chains(k=4, 6, 8)について書く.\\
%提案手法であるdouble-base chains(k=2)について, 実装法などを書く.
\section{提案手法のアルゴリズム}
BKLS Algorithm, Window Miller Algorithmを組みわせることで新しい高速化手法を提案する。以下にそのアルゴリズムを示す.
\par
\begin{table}[htbp]
 \begin{center}
  \begin{tabular}{|l|}
     \hline
     Input: $n, \ P, Q \in E(\mathbb{F}_q)[n]$ \\
     Output: $f \in \mathbb{F}_{q^k}$  \\
     \hline
     (online computation) \\
     1: \quad $P_1 = P, f'_1=1 $\\
     2: \quad for $i \gets 2$ up to do $2^w -1$\\
     3: \quad \quad $P_i \gets P + P_{i-1} $\\
     4: \quad \quad $f'_i \gets f'_{i-1} \cdot g_{P_i,\ P}(\psi (Q))$\\

     (main computation) \\
     5: \quad $V \gets P, f \gets 1 $\\
     6: \quad $n=\sum^{l - 1}_{i=0} n_i 2^i, \ n_i \in \{0,1\},\ n_0 = 1$\\

     7: \quad for $ n-1 \gets i$ down to 0 step w\\
     8: \quad step 8-1 から 8-2 をw回繰り返す\\
     8-1: \quad \quad $V \gets 2V $\\
     8-2: \quad \quad $f \gets f^2 \cdot g_{V,\ V}(\psi (Q))$\\

     9: \quad $n' \gets =\sum^{j=i-w+1}_{i} n_{j}2^{j-i+w-1} $\\
     10: \quad if $n' \neq 0$ then\\
     10-1: \quad \quad $f \gets ff'_m \cdot g_{V,\ P_m}(\psi (Q))$ \\
     10-2: \quad \quad $V \gets V + P_{n'} $\\
     11: \quad return $f$\\
     \hline
  \end{tabular}
 \end{center}
 \caption{Window方を用いたBKLS Algorithm}
\end{table}
\par
\clearpage

\section{提案手法の計算量}
Miller Algorithm, BKLS Algorithm, Window Miller Algorithm, 提案手法のアルゴリズムの計算量を比較する。
\par
Miller Algorithmにおける演算部分のステップを加算(TADD), 2倍算(TDBL), BKLS Algoritmにおけるに演算部分のステップを加算(BKLS-TADD), 2倍算(BKLS-TDBL)に分ける. \cite{TATE}によると、それぞれの計算コストは次のようになっている.
\begin{table}[htbp]
 \begin{center}
  \begin{tabular}{|l|c|}
  \hline
  演算 & 計算コスト \\
  \hline
  TADD & $4M_k + (6k+10)M + 4S$ \\
  \hline
  TDBL & $4M_k + 2S_k + (6k+7)M + 7S$ \\
  \hline
  BKLS-TADD & $M_k + (3k+10)M + 3S$ \\
  \hline
  BKLS-TDBL & $M_k + S_k + (3k+7)M + 6S$ \\
  \hline
  \end{tabular}
 \end{center}
 \caption{各演算の計算コスト}
\end{table}
\par
ただし、$\mathbb F_q$ 上の乗算、自乗の演算量を $M$, $S$, $\mathbb F_{q^k}$ 上の乗算、自乗の演算量を $M_k$, $S_k$ であらわしている。$k$ が比較的小さい値 (supersingular な楕円曲線の場合、$k = 2, 3, 6$) であることを考えると、ほぼ $M_k = k^2M$ と考えられる。ここで、 $S = M$ とすると \\

\begin{table}[htbp]
\begin{center}
\begin{tabular}{|l|c|}
\hline
 演算 & 計算コスト\\
 \hline
 TADD & $(4k^2 + 6k + 14)M$ \\
 \hline
 TDBL & $(6k^2 + 6k + 14)M$ \\
 \hline
  BKLS-TADD & $(k^2 + 3k + 13)M$ \\
  \hline
  BKLS-TDBL & $(2k^2 + 3k + 13)M$ \\
  \hline
\end{tabular}
\end{center}
\caption{計算量(近似)}
\end{table}

$n$を$l$ビットのランダムな数とし,これらを用いて Miller Algorithm, BKLS Algorithm, Window Miller Algorithm, 提案手法のアルゴリズムの計算量は次のように表される.
\begin{table}[htbp]
 \begin{center}
  \begin{tabular}{|l|c|c|c|}
  \hline
  Algorithm & 計算コスト \\
  \hline
  Miller Algorithm & $l$ \times $TDBL$ + $(l/2)$ \times $TADD$ \\
  \hline
  Window Miller Algorithm & $l$ \times $TDBL$ + $(l/w +2^w-2)$ \times $TADD$ \\
  \hline
  BKLS Algorithm & $l$ \times $BKLS-TDBL$ + $(l/2)$ \times $BKLS-TADD$  \\
  \hline
  提案手法 & $l$ \times $BKLS-TDBL$ + $(l/w +2^w-2)$ \times $BKLS-TADD$ \\
  \hline
  \end{tabular}
 \end{center}
 \caption{各手法の計算コスト}
\end{table}

\par
\newpage
