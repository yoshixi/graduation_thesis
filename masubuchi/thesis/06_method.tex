\chapter{既存研究と提案手法}
%既存手法であるwindow法, NAF, left-to-right, double-base chains(k=4, 6, 8)について書く.\\
%提案手法であるdouble-base chains(k=2)について, 実装法などを書く.
\section{Double-Base Chains}
% 楕円曲線$E(\mathbb{F}_q)$上の$k$倍点を求めるために2進展開, 3進展開, NAF, window法, フロベニウス展開などがある.
% Double-Base Number SystemはV. S. Dimitrovらによって提案された手法\cite{DBNSI}で, 整数$k$は2, 3のべき乗を使って次のように表すことができる.
% \par
% \[
% k = \sum ^m_{i = 1} s_i 2^{b_i} 3^{t_i},\ s_i \in _\{-1,1\},\ b_i,\ t_i \ge 0.
% \]
% \par
% このように$k$を展開する方法をDBNS展開と呼び \cite{DBNSI} , で指数計算の手法として提案された. この手法を楕円曲線$E(\mathbb{F}_q)$上の$k$倍点を求めるために適用する. この手法をDouble-Base Chains \cite{DBNS}と呼び, 以下にそのアルゴリズムを示す.
% \par
% \begin{table}[htbp]
%  \begin{center}
%   \begin{tabular}{|l|}
%      \hline
%      Input: $k,\ b_{max},\ t_{max} > 0$\\
%      Output: $k = \sum ^m_{i = 1} s_i 2^{b_i} 3^{t_i}となるような集合(s_i,b_i,t_i),\ b_1 \ge \cdots \ge b_m \ge 0,\ t_1 \ge \cdots \ge t_m \ge 0$\\
%      \hline
%      1. \quad $s \gets 1$\\
%      2. \quad while $k > 0$ do\\
%      3. \quad \quad $k$に最も近似した値$z = 2^b 3^t$を定義する. $0 \le b \le b_{max},\ 0 \le t \le t_{max}$\\
%      4. \quad \quad print $(s,b,t)$\\
%      5. \quad \quad $b_{max} \gets b,\ t_{max} \gets t$\\
%      6. \quad \quad if $k < z$ then\\
%      7. \quad \quad \quad $s \gets -s$\\
%      8. \quad \quad $k \gets |k - z|$\\
%      \hline
%   \end{tabular}
%  \end{center}
%  \caption{DBNS representation algorithm(Double-Base Chains)}
% \end{table}
% \par
% このとき, 2進展開と3進展開の上限をそれぞれ$b_{max},\ t_{max}$とし, $m$はループ回数を表す. これらの値は$b_{max} < \mbox{log} _2(k) < n,\ t_{max} < \mbox{log} _3(k) \approx 0.63n$となる. このアルゴリズムの複雑性はStep 3に依存する. しかし, これは事前計算した値のテーブルを使うことで容易に解決できる.
% さらにここで詳しくは述べないが, Double-Base Chainsは暗号に対する攻撃手法の一つであるサイドチャネル攻撃に耐性がある.
% \clearpage
% \section{Double-Base Chainsを用いたMiller Algorithm}
% Miller Algorithm, Signed Miller Algorithmでそれぞれ部分群の位数$n$に対して2進展開, 符号付き2進展開が使われていた. 次にC. Zhaoらによって提案されたDouble-Base Chainsを用いたMiller Algorithm\cite{DBCT}を次に示す.
% \begin{table}[htbp]
%  \begin{center}
%   \begin{tabular}{|l|}
%      \hline
%      Input: $n,\ P = (x_P,y_P) \in E(\mathbb{F}_q)[n],\ Q = (x_Q,y_Q) \in E(\mathbb{F}_{q^k})[n]$\\
%      Output: $e_n(P,Q)$\\
%      \hline
%      1. \quad $T \gets P,\ f_{-1} = \frac{1}{x_Q - x_P}$\\
%      2. \quad if $s_1 = -1$ then
%      3. \quad \quad $f_1 \gets f_{-1}$\\
%      4. \quad $n = \sum ^m_{i = 1} s_i 2^{b_i} 3^{t_i},\ s_i \in \{-1,1\},\ b_1 \ge b_2 \ge \cdots \ge b_m \ge 0,\ t_1 \ge t_2 \ge \cdots \ge t_m \ge 0$\\
%      5. \quad for $i = 1,\ \cdots ,\ m - 1$ do\\
%      6. \quad \quad $u \gets b_i - b_{i + 1},\ v \gets t_i - t_{i + 1}$\\
%      7. \quad \quad if $u = 0$ then\\
%      8. \quad \quad \quad for $j = 1,\ \cdots ,\ v$ do\\
%      9. \quad \quad \quad \quad $f_1 \gets f_1^3 \cdot \frac{g_{T,T}(Q)g_{T,2T}}{g_{2T}(Q)g_{3T}(Q)},\ T \gets 3T$\\
%      10. \quad \quad else\\
%      11. \quad \quad \quad if $v = 0$ then\\
%      12. \quad \quad \quad \quad for $j = 1,\ \cdots ,\ u$ do\\
%      13. \quad \quad \quad \quad \quad $f_1 \gets f_1^2 \cdot \frac{g_{T,T}(Q)}{g_{2T}(Q)},\ T \gets 2T$\\
%      14. \quad \quad \quad else\\
%      15. \quad \quad \quad \quad for $j = 1,\ \cdots ,\ u$ do\\
%      16. \quad \quad \quad \quad \quad $f_1 \gets f_1^2 \cdot \frac{g_{T,T}(Q)}{g_{2T}(Q)},\ T \gets 2T$\\
%      17. \quad \quad \quad \quad for $j = 1,\ \cdots ,\ v$ do\\
%      18. \quad \quad \quad \quad \quad $f_1 \gets f_1^3 \cdot \frac{g_{T,T}(Q)g_{T,2T}}{g_{2T}(Q)g_{3T}(Q)},\ T \gets 3T$\\
%      19. \quad if $s_{i + 1} = 1$ then\\
%      20. \quad \quad $f_1 = f_1 \cdot \frac{g_{T,P}(Q)}{g_{T + P}(Q)},\ T \gets T + P$\\
%      21. \quad else\\
%      22. \quad \quad $f_1 = f_1 \cdot f_{-1} \cdot \frac{g_{T,-P}}{g_{T - P}},\ T \gets T - P$\\
%      23. \quad return $f$\\
%      \hline
%   \end{tabular}
%  \end{center}
%  \caption{Double-Base Chainsを用いたMiller Algorithm}
% \end{table}
% \clearpage
% \par
% C. Zhaoらは埋め込み次数$k=4,6,8$のとき, 上記の手法とMiller Algorithm, Signed Miller Algorithmの計算コストを比較した.
% 計算コストを求めるにあたってC. Zhaoらは\cite{ECT}を参照した. 計算コストは$\mathbb{F}_q^\ast$上では乗算一回の演算時間を$1M$, 2乗を$1S$, 除算を$1I$とする. $\mathbb{F}_{q^k}^\ast$では乗算一回の演算時間を$1M_k$, 2乗を$1S_k$, 除算を$1I_k$とし, $\mathbb{F}_q^\ast と\mathbb{F}_{q^k}^\ast$の要素の乗算一回の演算時間を$1M_b$とする. 同一の演算, スカラー倍の演算は時間がかからないものとする. また, $\mathbb{F}_q^\ast,\ \mathbb{F}_{q^k}^\ast$上の加算, 減算演算時間は乗算のコストに比べて小さいので無視する. さらに, $S = 0.8M,\ I = 10M,\ M_k = k^{1.6}M,\ S_k = 0.8k^{1.6}M,\ I_k = 10k^{1.6}M,\ M_b = kM$と換算できる.
% 楕円曲線$E(\mathbb{F}_q)$上では点の加算(ECADD), 2倍算(ECDBL), 3倍算(ECTRL)がある. 減算は加算と同様のコストである. それぞれの計算コストは次のようになっている.
% \begin{table}[htbp]
%  \begin{center}
%   \begin{tabular}{|l|c|}
%   \hline
%   演算 & 計算コスト \\
%   \hline
%   ECADD & $1I + 2M + 1S$ \\
%   \hline
%   ECDBL & $1I + 2M + 2S$ \\
%   \hline
%   ECTRL & $1I + 7M + 4S$ \\
%   \hline
%   \end{tabular}
%  \end{center}
%  \caption{楕円曲線における各演算の計算コスト}
% \end{table}
% \par
% Tate ペアリングにおける演算部分のステップを加算(TADD), 減算(TSUB), 2倍算(TDBL), 3倍算(TTRL)に分ける. それぞれの計算コストは次のようになっている.
% \begin{table}[htbp]
%  \begin{center}
%   \begin{tabular}{|l|c|}
%   \hline
%   演算 & 計算コスト \\
%   \hline
%   TADD & $M_k + 2.5M_b + 1I + 3M + 1S$ \\
%   \hline
%   TSUB & $M_k + 1I + (2k + 3)M + 1S$ \\
%   \hline
%   TDBL & $M_k + S_k + 3.5M_b + 1I + 4M + 2S$ \\
%   \hline
%   TTRL & $3M_k + S_k + 2M_b + 1I + 9M + 4S$ \\
%   \hline
%   \end{tabular}
%  \end{center}
%  \caption{Tate ペアリングにおける各演算の計算コスト}
% \end{table}
% \par
% $n$のbit数を$l$とし, これらを用いてMiller Algorithm, Signed Miller Algorithm, Double-Base Chainsを用いたMiller Algorithmの計算コストは次のように表される. \\
% %Miller Algorithm
% %\[
% %(3n - 2)M_k + (2n - 2)S_k + (\frac{k}{2}(9n - 6) + 12n - 7) + (15n -12)\frac{s}{2}
% %\]
% %Signed Miller Algorithm
% %\[
% %(\frac{4}{3}n + 5)M_k + (l - 1)S_k + I_k + 7M_{\frac{k}{2}} + I_{\frac{k}{2}} + (\frac{17n - 14}{4}k + (5n - 4)) + (\frac{7}{3}n - 2)s+ (\frac{4}{3}n - 1)I
% %\]
% %Double-Base Chainsを用いたMiller Algorithm
% %\[
% %(b_{max} + 3t_{max} + m + 6)M_k + (b_{max} + t_{max} + 1)S_k + (\frac{7}{2}b_{max} + 2t_{max} + \frac{5}{4} m)M_b
% %\]
% %\[
% % + (b_{max} + t_{max} + m)I + (4b_{max} + 9t_{max} + (k + 3)m) + (2b_{max} + 4t_{max} + m)s + I_k + 7M_{\frac{k}{2}} + I_{\frac{k}{2}}
% %\]
% \begin{table}[htbp]
% \begin{center}
% \begin{tabular}{|l|c|}
% \hline
%  & 計算コスト\\
% \hline
% \raisebox{1.5ex}[0cm][0cm]{Miller Algorithm} & \shortstack{$(3l - 2)M_k + (2l - 2)S_k + (\frac{9l - 6}{2}k + 12l - 7) $\\$+ (15l -12)\frac{s}{2}$}\\
% \hline
% \raisebox{1.5ex}[0cm][0cm]{Signed Miller Algorithm} & \shortstack{$(\frac{4}{3}l + 5)M_k + (l - 1)S_k + I_k + 7M_{\frac{k}{2}} + I_{\frac{k}{2}} $\\$+ (\frac{17l - 14}{4}k + (5l - 4)) + (\frac{7}{3}l - 2)s+ (\frac{4}{3}l - 1)I$}\\
% \hline
% \raisebox{4.5ex}[0cm][0cm]{Double-Base Chainsを用いたMiller Algorithm} & \shortstack{$(b_{max} + 3t_{max} + m + 6)M_k + (b_{max} + t_{max} + 1)S_k $\\$+ (\frac{7}{2}b_{max} + 2t_{max} + \frac{5}{4} m)M_b + (b_{max} + t_{max} + m)I $\\$+ (4b_{max} + 9t_{max} + (k + 3)m) + (2b_{max} + 4t_{max} + m)s $\\$+ I_k + 7M_{\frac{k}{2}} + I_{\frac{k}{2}}$}\\
% \hline
%   \end{tabular}
%  \end{center}
%  \caption{各手法の計算コスト}
% \end{table}
% \clearpage
% 次にC. Zhaoらによって示されたこの手法の$k=4,6,8$のときの計算コストと従来法を$b_{max} = 76,\ t_{max} = 53,\ m = 38$のときの計算コストで比較したものを表にした. 表の計算コストは乗算に換算している.
% \begin{table}[htbp]
%  \begin{center}
%   \begin{tabular}{|c|c|c|c|c|c|}
%   \hline
%   $b_{max}$ & $t_{max}$ & $m$ & cost($k=4$) & cost($k=6$) & cost($k=8$) \\
%   \hline
%   57 & 65 & 45 & 8287 & 12744 & 17222 \\
%   \hline
%   76 & 53 & 38 & 8350 & 12554 & 17085 \\
%   \hline
%   95 & 41 & 37 & 8395 & 12552 & 17052 \\
%   \hline
%   103 & 36 & 39 & 8493 & 12676 & 17186 \\
%   \hline
%   \end{tabular}
%  \end{center}
%  \caption{$k=4,\ 6,\ 8$におけるDouble-Base Chainsを用いたMiller Algorithmの計算コスト}
% \end{table}
% \begin{table}[htbp]
%  \begin{center}
%   \begin{tabular}{|l|c|c|c|}
%   \hline
%   Algorithm & cost($k=4$) & cost($k=6$) & cost($k=8$) \\
%   \hline
%   Miller Algorihtm & 12328 & 20353 & 28379 \\
%   \hline
%   Signed Miller Algorithm & 9196 & 13685 & 18121 \\
%   \hline
%   Double-Base Chainsを用いたMiller Algorithm & 8350 & 12554 & 17085 \\
%   \hline
%   \end{tabular}
%  \end{center}
%  \caption{$k=4,\ 6,\ 8$における各手法の計算コスト}
% \end{table}
% \section{提案手法}
% 埋め込み次数$k=2$について, Miller Algorithm, Signed Miller Algorithm, Double-Base Chainsを用いたMiller Algorithmの計算コストを算出する. そして提案手法である$k=2$に加えて既存研究である$k=4,\ 6$の場合をMAGMAで実装し, 計算時間を比較した. さらに考察では結果を踏まえた上で, Double-Base Chainsを用いたMiller Algorithmの新たな高速化手法をいくつか提案する. 
