% \chapter{実行結果}
% \section{実行結果}
% \subsection{実装環境}
% \noindent プロセッサ: Intel(R) Xeon(TM) CPU 3.40GHz\\
% メモリ: 2046MB RAM\\
% OS: Microsoft Windows XP Professional\\
% プログラミング言語: Magma ver.2.14-6
%
% \subsection{実装条件}
% \noindent

% 部分群の位数$n$はランダムに選ぶ. \\
% 計測は演算部分だけに対して1000回行い, その平均を取った. \\
% 楕円曲線は埋め込み次数$k=2,\ 4,\ 6$でordinary curveを使用した. \\
% 実験で使用した曲線は以下の通り. $k=4$の曲線は\cite{OPM}, $k=6$の曲線は\cite{GMM}の例題から引用した.
% \begin{enumerate}
% \item $k=2のとき,\ y^2=x^3+x$\\
% \item $k=4のときy^2 = x^3 - 3x + b,\ b=2993400528565252484131025435851646783754072514402$\\
% \item $k=6のときy^2 = x^3 - 3x + b,\ b=1067782606939229981648974648369145174879546988730$\\
% \end{enumerate}
% \clearpage
% \subsection{実行結果}
% $\mbox{log}n \approx 80,120,160$のときの従来手法と提案手法の計算コストと計算時間を次の表に示す. \\
% %\vspace{7mm}
% \begin{table}[htb]
%  \begin{center}
%   \begin{tabular}{|l|c|c|}
%   \hline
%   82bit($b_{max} = 50,\ t_{max} = 20,\ m = 13$) & 計算コスト& 計算時間\\
%   \hline
%   Miller Algorithm & 3329 & 12.9ms \\
%   \hline
%   Signed Miller Algorithm & 2906 & 11.3ms \\
%   \hline
%   Double-Base Chainsを用いたMiller Algorithm & 2487 & 9.6ms \\
%   \hline
%   \end{tabular}
%  \end{center}
%  \caption{$k=2,\ \mbox{log}n \approx 80$}
% \end{table}
% \begin{table}[htb]
%  \begin{center}
%   \begin{tabular}{|l|c|c|}
%   \hline
%   120bit($b_{max} = 68,\ t_{max} = 33,\ m = 13$) & 計算コスト& 計算時間\\
%   \hline
%   Miller Algorithm & 4885 & 21.1ms \\
%   \hline
%   Signed Miller Algorithm & 4242 & 18.6ms \\
%   \hline
%   Double-Base Chainsを用いたMiller Algorithm & 3483 & 15.5ms \\
%   \hline
%   \end{tabular}
%  \end{center}
%  \caption{$k=2,\ \mbox{log}n \approx 120$}
% \end{table}
% \begin{table}[htb]
%  \begin{center}
%   \begin{tabular}{|l|c|c|}
%   \hline
%   159bit($b_{max} = 97,\ t_{max} = 39,\ m = 17$) & 計算コスト& 計算時間\\
%   \hline
%   Miller Algorithm & 6481 & 32.9ms \\
%   \hline
%   Signed Miller Algorithm & 5614 & 29.2ms \\
%   \hline
%   Double-Base Chainsを用いたMiller Algorithm & 4565 & 25.1ms \\
%   \hline
%   \end{tabular}
%  \end{center}
%  \caption{$k=2,\ \mbox{log}n \approx 160$}
% \end{table}
% \begin{table}[htb]
%  \begin{center}
%   \begin{tabular}{|l|c|c|}
%   \hline
%   161bit($b_{max} = 72,\ t_{max} = 56,\ m = 24$) & 計算コスト& 計算時間\\
%   \hline
%   Miller Algorithm & 6563 & 43.8ms \\
%   \hline
%   Signed Miller Algorithm & 5684 & 42.5ms \\
%   \hline
%   Double-Base Chainsを用いたMiller Algorithm & 4697 & 34.9ms \\
%   \hline
%   \end{tabular}
%  \end{center}
%  \caption{$k=2,\ \mbox{log}n \approx 160$}
% \end{table}
% \begin{table}[htb]
%  \begin{center}
%   \begin{tabular}{|l|c|c|}
%   \hline
%   162bit($b_{max} = 113,\ t_{max} = 31,\ m = 45$) & 計算コスト& 計算時間\\
%   \hline
%   Miller Algorithm & 12623 & 66.2ms \\
%   \hline
%   Signed Miller Algorithm & 9273 & 64.0ms \\
%   \hline
%   Double-Base Chainsを用いたMiller Algorithm & 8796 & 59.0ms \\
%   \hline
%   \end{tabular}
%  \end{center}
%  \caption{$k=4,\ \mbox{log}n \approx 160$}
% \end{table}
% \begin{table}[htbp]
%  \begin{center}
%   \begin{tabular}{|l|c|c|}
%   \hline
%   159bit($b_{max} = 80,\ t_{max} = 49,\ m = 36$) & 計算コスト& 計算時間\\
%   \hline
%   Miller Algorithm & 19795 & 78.4ms \\
%   \hline
%   Signed Miller Algorithm & 13289 & 77.8ms \\
%   \hline
%   Double-Base Chainsを用いたMiller Algorithm & 12205 & 64.8ms \\
%   \hline
%   \end{tabular}
%  \end{center}
%  \caption{$k=6,\ \mbox{log}n \approx 160$}
% \end{table}
% \vspace{2cm}
% \par
% 提案手法は従来手法に比べて表7.1から82bitでは34\%, 18\%, 表7.2から120bitでは36\%, 20\%, 表7.3から159bitでは31\%, 16\%, 表7.4から161bitでは25\%, 22\%の高速化を実現した.
% さらに既存研究を実装した結果、表7.5から$k=4$で160bitのとき,  12\%, 8\%, 表7.6から$k=6$で160bitのとき, 21\%, 20\%の高速化を実現した.
% \clearpage

\chapter{評価と考察}
\section{評価}
l \approx $80,120,160$ , $k=2, 4$ それぞれの場合に応じて,従来手法と提案手法の計算コストと計算時間を次の表に示す.

\begin{table}[htbp]
 \begin{center}
  \begin{tabular}{|l|c|c|c|}
  \hline
  $k=2$,$l=80$, $w=4$ & 計算コスト \\
  \hline
  Miller Algorihtm & 5980M \\
  \hline
  Window Miller Algorithm & 5248M \\
  \hline
  BKLS Algorithm &  3080M \\
  \hline
  提案手法 & 2942M \\
  \hline
  \end{tabular}
 \end{center}
 \caption{各手法の計算コスト}
\end{table}

\begin{table}[htbp]
 \begin{center}
  \begin{tabular}{|l|c|c|c|}
  \hline

  $k=6$,$l=80$, $w=4$ & 計算コスト \\
  \hline
  Miller Algorihtm & 29040M \\
  \hline
  Window Miller Algorithm & 27876M \\
  \hline
  BKLS Algorithm &  10920M \\
  \hline
  提案手法 & 10518M \\
  \hline
  \end{tabular}
 \end{center}
 \caption{各手法の計算コスト}
\end{table}

\begin{table}[htbp]
 \begin{center}
  \begin{tabular}{|l|c|c|c|}
  \hline
  $k=2$,$l=120$, $w=4$ & 計算コスト \\
  \hline
  Miller Algorihtm & 8520M \\
  \hline
  Window Miller Algorithm & 7848M \\
  \hline
  BKLS Algorithm &  4620M \\
  \hline
  提案手法 & 4250M \\
  \hline
  \end{tabular}
 \end{center}
 \caption{各手法の計算コスト}
\end{table}

\clearpage

\begin{table}[htbp]
 \begin{center}
  \begin{tabular}{|l|c|c|c|}
  \hline
  $k=6$,$l=120$, $w=4$ & 計算コスト \\
  \hline
  Miller Algorihtm & 43560M \\
  \hline
  Window Miller Algorithm & 40456M \\
  \hline
  BKLS Algorithm &  16380M \\
  \hline
  提案手法 & 15308M \\
  \hline
  \end{tabular}
 \end{center}
 \caption{各手法の計算コスト}
\end{table}

\begin{table}[htbp]
 \begin{center}
  \begin{tabular}{|l|c|c|c|}
  \hline
  $k=2$,$l=160$, $w=4$ & 計算コスト \\
  \hline
  Miller Algorihtm & 11360M \\
  \hline
  Window Miller Algorithm & 10268M \\
  \hline
  BKLS Algorithm &  6160M \\
  \hline
  提案手法 & 5562M \\
  \hline
  \end{tabular}
 \end{center}
 \caption{各手法の計算コスト}
\end{table}

\begin{table}[htbp]
 \begin{center}
  \begin{tabular}{|l|c|c|c|}
  \hline
  $k=6$,$l=160$, $w=4$ & 計算コスト \\
  \hline
  Miller Algorihtm & 58080M \\
  \hline
  Window Miller Algorithm & 53036M \\
  \hline
  BKLS Algorithm &  21840M \\
  \hline
  提案手法 & 20098M \\
  \hline
  \end{tabular}
 \end{center}
 \caption{各手法の計算コスト}
\end{table}

\par
\par
\section{考察}

提案手法は従来のBKLS Algorithmに比べて表 7.1, 表 7.2 から 80bit では 4.5\%, 3.7\%, 表7.3, 表7.4 から 120bit では 8.0\%, 6.5\%, 表7.5, 表7.6 から 160bit では 9.7\%, 8\%の高速化を実現できる。
ペアリングで使用する2つの点の位数$n$が大きいほど、高速化できる割合は大きく、埋め込み次数$k$が大きいほど高速化できる割合は小さくなることが分かる。また、今回はwの値を動かさなかったが、$w$ の値の設定によって異なる結果が出ることが予想される。
\par
位数$n$が小さく、埋め込み字数$k$が大きいほど、今回提案した手法のメリットを享受できなくなる。提案した手法は、事前演算を必要とするため、場合によって使い分ける必要がある。

\section{今後の課題}
今後の課題としては,既存のSigned Miller Algorithmや, 3倍算の計算コストが乗算よりも小さいことを利用した標数3の有限体におけるMiller Algorithmにdistortion mapを適用した手法に今回の提案に使用したWindow Miller Algorithmを組み合わせるなどが考えられる。また、今回の手法を適用するのに適した楕円曲線を探すなどが挙げられる。
